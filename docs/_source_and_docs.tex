The following was generated from a Jupyter notebook with the following
filename and kernel.

\begin{minted}[]{matlab}
disp(['notebook filename: ',the_notebook,'.ipynb'])
disp(['notebook kernel: ',the_kernel])
\end{minted}

\begin{minted}[]{text}
notebook filename: _source_and_docs.ipynb
notebook kernel: matlab
\end{minted}

\hypertarget{tf_factor-docs}{%
\section{\texorpdfstring{\mintinline[]{text}{tf_factor}
docs}{ docs}}\label{tf_factor-docs}}

The following is the source code. Executing the cell writes the source
file!

\begin{minted}[]{matlab}
%%file ../tf_factor.m
function out = tf_factor(sys)
% TF_FACTOR  factors a transfer function TF object
%   SYS_ARRAY = TF_FACTOR(SYS) factors SYS into
%   constant, real pole/zero, and 
%   conjugate pole/zero pair sub-TF models. 
%   It returns a TF model array. 
%   The last entry is the appropriate gain.
%   The product of entries of the model array 
%   should equal sys.
%   
%   Dependencies:
%     - matlab-rico functions
%       - POLE 
%       - ZERO
%     - toolboxes
%       - Control Systems
%
%   Example:
%
%   sys=tf(...
%     [-0.64 -0.4101 0.00783],...
%     [1 1.489 0.7681 0.09455 0.0424 .7]...
%   );
%   tf_factor(sys)
%   
%   source: https://github.com/ricopicone/matlab-rico
%   
%   See also TF, STACK.

if ~isa(sys,'tf')
  sys = tf(sys);
end

% extract poles and zeros
poles=pole(sys);
zeros=zero(sys);

% make sure they're in coupled pairs
poles_cplx = cplxpair(poles);
zeros_cplx = cplxpair(zeros);

% loop through and extract sub-tfs into model array, each in standard form, keeping track of the gain
F = stack(1,tf(1,1)); % init model array
F_gain = sys.Num{:}(...
  find(cell2mat(sys.Num),1,'first')...
); % overall gain of stack
F_gain_o = F_gain ; % original gain
k=1;
jskip=0;
% poles first
for j=1:length(poles_cplx)
  if ~jskip
    if ~isreal(poles_cplx(j))
      F(:,:,k) = zpk([],[poles_cplx(j),poles_cplx(j+1)],poles_cplx(j)*poles_cplx(j+1));
      F_gain = F_gain/abs(poles_cplx(j)*poles_cplx(j+1));
      jskip=1;% skip next index
    else
      F(:,:,k) = zpk([],poles_cplx(j),abs(poles_cplx(j)));
      F_gain = F_gain/abs(poles_cplx(j));
      jskip=0;
    end
    k=k+1;
  else
    jskip=0;
  end
end
% now zeros
for j=1:length(zeros_cplx)
  if ~jskip
    if ~isreal(zeros_cplx(j))
      F(:,:,k) = zpk([zeros_cplx(j),zeros_cplx(j+1)],[],1/(zeros_cplx(j)*zeros_cplx(j+1)));
      F_gain = F_gain*abs(zeros_cplx(j)*zeros_cplx(j+1));
      jskip=1;% skip next index
    else
      F(:,:,k) = zpk(zeros_cplx(j),[],1/abs(zeros_cplx(j)));
      F_gain = F_gain*abs(zeros_cplx(j));
      jskip=0;
    end
    k=k+1;
  else
    jskip=0;
  end
end
F(:,:,k) = F_gain; % drop the overall gain into the model array

% check by concatenation
tf_composite = 1;
for j=1:k
  tf_composite = tf_composite*F(:,:,j);
end

if (... % check that the factorization is correct
  isequal(sys.Num{:}(find(cell2mat(tf_composite.Num),1,'first')),F_gain_o) && ... % gain
  round(sum(poles),5) == round(sum(pole(tf_composite)),5) && ... % poles ... not perfect
  round(sum(zeros),5) == round(sum(zero(tf_composite)),5) ... % zeros ... not perfect
)
  out = F;
else
  error('composite check failed!')
  out = 1;
end
\end{minted}

\begin{minted}[]{text}
Created file '/Users/picone/code/matlab-rico/tf_factor.m'.
\end{minted}

\hypertarget{usage-and-examples}{%
\subsection{Usage and examples}\label{usage-and-examples}}

\hypertarget{bode_multi-docs}{%
\section{\texorpdfstring{\mintinline[]{text}{bode_multi}
docs}{ docs}}\label{bode_multi-docs}}

The following is the source code. Executing the cell writes the source
file!

\begin{minted}[]{matlab}
%%file ../bode_multi.m
function [out,ax1,ax2] = bode_multi(sys_a)

syms s

if ~isa(sys_a,'tf')
  sys_a = tf(sys_a);
end
n = length(sys_a); % > 1 if system model array

out = figure;
ax1 = subplot(2,1,1);
ax2 = subplot(2,1,2);
omega_a = {.1,1};
mag_lims = [0,1];
phase_lims = [-90,0];
for i = 1:n
  sys = sys_a(1,1,i);
  [mag,phase,omega] = bode(sys);
  if omega(1) < omega_a{1}
    omega_a{1} = omega(1);
  end
  if omega(end) > omega_a{end}
    omega_a{end} = omega(end);
  end
  if mag_lims(1) < mag_lims(1)
    mag_lims(1) = mag_lims(1);
  end
  if mag_lims(2) > mag_lims(2)
    mag_lims(2) = mag_lims(2);
  end
  if phase_lims(1) < phase_lims(1)
    phase_lims(1) = phase_lims(1);
  end
  if phase_lims(2) > phase_lims(2)
    phase_lims(2) = phase_lims(2);
  end
end

olog = num2cell(cellfun(@(x) log10(x),omega_a));
omega = logspace(olog{:},100);

for i = 1:n
  sys = sys_a(1,1,i);
  [mag,phase] = bode(sys,omega);
  mag = squeeze(mag);
  phase = squeeze(phase);
  % size(omega)
  % omega = squeeze(omega);
  axes(ax1);
  hold on;
  [num,den] = tfdata(sys);
  sys_sym = poly2sym(cell2mat(num),s)/poly2sym(cell2mat(den),s);
  semilogx(...
    omega,db(mag),...
    'linewidth',1,...
    'displayname',['$',latex(sys_sym),'$']...
  );
  ylabel('|H(j\omega)|, dB')
  axes(ax2);
  hold on;
  semilogx(omega,phase,'linewidth',1);
  xlabel('frequency \omega, rad/s')
  ylabel('\angle{H(j\omega)}, deg')
  h = findobj(gcf,'type','line');
  set(h,'linewidth',1);
end
% log scale
ax1.XScale = 'log';
ax2.XScale = 'log';
% adjust limits and ticks
mag_tick_array = ax1.YLim(1):20:ax1.YLim(2);
[m0db,i0db_a] = min(abs(mag_tick_array));
i0db = i0db_a(1); % first index closest to zero
mag_tick_array = mag_tick_array-mag_tick_array(i0db);
ax1.YTick = mag_tick_array;
phase_tick_array = ax2.YLim(1):45:ax2.YLim(2);
[p0db,i0_a] = min(abs(phase_tick_array));
i0 = i0_a(1); % first index closest to zero
phase_tick_array = phase_tick_array-phase_tick_array(i0);
ax2.YTick = phase_tick_array;
% grid lines
ax1.XGrid = 'on';
ax1.YGrid = 'on';
ax2.XGrid = 'on';
ax2.YGrid = 'on';
% legend
axP = get(ax1,'Position'); % so we can keep size
l = legend(ax1,'show');
l.Interpreter = 'latex';
l.Location = 'northeastoutside';
ax1.Position = axP; % reset size
\end{minted}

\begin{minted}[]{text}
Created file '/Users/picone/code/matlab-rico/bode_multi.m'.
\end{minted}

\hypertarget{usage-and-examples-1}{%
\subsection{Usage and examples}\label{usage-and-examples-1}}

\begin{minted}[]{matlab}
sys = tf([5,3,5],[1,6,1,20])
\end{minted}

\begin{minted}[]{text}
sys =
 
    5 s^2 + 3 s + 5
  --------------------
  s^3 + 6 s^2 + s + 20
 
Continuous-time transfer function.
\end{minted}

\begin{minted}[]{matlab}
sys_a = tf_factor(sys)
\end{minted}

\begin{minted}[]{text}
sys_a(:,:,1,1) =
 
          3.155
  ----------------------
  s^2 - 0.3399 s + 3.155
 

sys_a(:,:,2,1) =
 
    6.34
  --------
  s + 6.34
 

sys_a(:,:,3,1) =
 
  s^2 + 0.6 s + 1
 

sys_a(:,:,4,1) =
 
  0.25
 
4x1 array of continuous-time transfer functions.
\end{minted}

\begin{minted}[]{matlab}
% [f,ax_mag,ax_phase] = bode_multi(G); % get axis handles
f = bode_multi(sys_a);

hgsave(f,'figures/temp');
\end{minted}

\begin{figure}
\centering
% This file was created by matlab2tikz.
%
\definecolor{mycolor1}{rgb}{0.00000,0.44700,0.74100}%
\definecolor{mycolor2}{rgb}{0.85000,0.32500,0.09800}%
\definecolor{mycolor3}{rgb}{0.92900,0.69400,0.12500}%
\definecolor{mycolor4}{rgb}{0.49400,0.18400,0.55600}%
%
\begin{tikzpicture}

\begin{axis}[%
width=0.856\linewidth,
height=0.209\linewidth,
at={(0\linewidth,0.291\linewidth)},
scale only axis,
xmode=log,
xmin=0.01,
xmax=1000,
xminorticks=true,
ymin=-150,
ymax=150,
ytick={-140, -120, -100,  -80,  -60,  -40,  -20,    0,   20,   40,   60,   80,  100,  120,  140,  160},
ylabel style={font=\color{white!15!black}},
ylabel={$\text{$|$H(j}\omega\text{)$|$, dB}$},
axis background/.style={fill=white},
axis x line*=bottom,
axis y line*=left,
xmajorgrids,
xminorgrids,
ymajorgrids,
legend style={at={(1.03,1)}, anchor=north west, legend cell align=left, align=left, draw=white!15!black},
scaled ticks=false,tick label style={/pgf/number format/fixed}
]
\addplot [color=mycolor1, line width=1.0pt]
  table[row sep=crcr]{%
0.01	0.000270300015415614\\
0.025353644939701	0.001737649503184\\
0.0403701725859658	0.00440622614098629\\
0.0572236765935013	0.00885531691591268\\
0.072208090183854	0.0141042682296302\\
0.0911162756115487	0.0224683872885976\\
0.102353102189903	0.0283611401736721\\
0.114975699539774	0.0358024440751024\\
0.129154966501489	0.0452010652798549\\
0.145082877849596	0.0570747579229192\\
0.162975083462067	0.0720799724271615\\
0.183073828029534	0.0910500506407743\\
0.205651230834863	0.115044564156676\\
0.231012970008314	0.145413420886769\\
0.259502421139972	0.183880765489903\\
0.291505306282517	0.232655768296127\\
0.327454916287773	0.294580526347147\\
0.367837977182865	0.37333015374827\\
0.413201240011537	0.473687878004\\
0.464158883361283	0.60193067892601\\
0.521400828799976	0.766382621910566\\
0.585702081805658	0.978231166148191\\
0.65793322465756	1.25277188968749\\
0.739072203352571	1.6113825651442\\
0.83021756813197	2.08480408847902\\
0.932603346883218	2.71890712951642\\
1.04761575278967	3.58552461043547\\
1.176811952435	4.80440599518982\\
1.32194114846604	6.5908167996692\\
1.48496826225448	9.34839303941823\\
1.66810053720008	13.3529302649777\\
1.87381742286036	12.7143632069603\\
2.104904144512	6.67514765311171\\
2.36448941264539	1.79632002944993\\
2.65608778294667	-2.06926105658493\\
2.98364724028333	-5.3436965162677\\
3.35160265093885	-8.25319886536948\\
3.76493580679249	-10.922855372804\\
4.22924287438953	-13.427281313901\\
4.75081016210285	-15.8139148494944\\
5.33669923120639	-18.1144403107742\\
5.99484250318932	-20.3508346227334\\
7.56463327554623	-24.6897688618075\\
9.54548456661833	-28.9127580837559\\
13.5304777457982	-35.1253748240357\\
21.5443469003186	-43.2961832018219\\
48.6260158006541	-57.484353367962\\
628.029144183417	-101.940113146117\\
1000	-110.020962527339\\
};
\addlegendentry{$\frac{7103629588677099}{2251799813685248\, \left(s^2 - \frac{6122320703535551\, s}{18014398509481984} + \frac{7103629588677099}{2251799813685248}\right)}$}

\addplot [color=mycolor2, line width=1.0pt]
  table[row sep=crcr]{%
0.01	-1.08050016365269e-05\\
0.102353102189903	-0.00113180290975379\\
0.183073828029537	-0.00361990268032741\\
0.259502421139972	-0.00727017134141761\\
0.327454916287773	-0.0115704368758429\\
0.413201240011537	-0.0184089000390486\\
0.464158883361275	-0.0232165330648968\\
0.521400828799967	-0.0292754928252066\\
0.585702081805667	-0.0369089895871184\\
0.657933224657571	-0.0465222621401153\\
0.739072203352584	-0.058622547901507\\
0.83021756813197	-0.0738434218022235\\
0.932603346883218	-0.0929741710357348\\
1.04761575278967	-0.116994796344727\\
1.176811952435	-0.147117014584637\\
1.32194114846602	-0.184831200481767\\
1.48496826225446	-0.231958439923346\\
1.66810053720006	-0.290705643621436\\
1.87381742286039	-0.363719859574132\\
2.10490414451203	-0.454135447030012\\
2.36448941264543	-0.565604699714129\\
2.65608778294667	-0.702299186176674\\
2.98364724028333	-0.868866325142392\\
3.35160265093885	-1.0703249293457\\
3.76493580679249	-1.31188651276295\\
4.22924287438953	-1.59869789566011\\
4.75081016210277	-1.93551575048048\\
5.3366992312063	-2.32634335712032\\
5.99484250318942	-2.77407873173598\\
6.73415065775086	-3.28023373846219\\
7.56463327554635	-3.84477886899788\\
8.4975343590864	-4.46614632901549\\
9.54548456661833	-5.14139076649093\\
10.7226722201033	-5.86647390933865\\
12.0450354025879	-6.6366179200337\\
13.5304777457982	-7.44666800536498\\
15.1991108295293	-8.29141572689213\\
17.0735264747069	-9.16585351242139\\
19.1791026167249	-10.0653503979472\\
24.2012826479436	-11.9234315149765\\
30.5385550883341	-13.8386334869165\\
38.5352859371055	-15.791570563453\\
54.6227721768434	-18.7640002132231\\
97.7009957299225	-23.7746391823356\\
312.571584968824	-33.8591868612698\\
1000	-43.9585852159569\\
};
\addlegendentry{$\frac{3569022242513357}{562949953421312\, \left(s + \frac{3569022242513357}{562949953421312}\right)}$}

\addplot [color=mycolor3, line width=1.0pt]
  table[row sep=crcr]{%
0.01	-0.000712257924078585\\
0.0200923300256502	-0.0028755810808434\\
0.0284803586843579	-0.00577820698657661\\
0.0359381366380464	-0.00920146773944452\\
0.0453487850812863	-0.0146536905681955\\
0.0572236765935013	-0.0233387761386439\\
0.0642807311728424	-0.029455481259788\\
0.072208090183854	-0.0371770102002529\\
0.0811130830789682	-0.0469254380222424\\
0.0911162756115487	-0.059234444619733\\
0.102353102189903	-0.0747791706700696\\
0.114975699539774	-0.0944142558906265\\
0.129154966501489	-0.119222407473444\\
0.145082877849596	-0.150576571208433\\
0.162975083462067	-0.190219743522221\\
0.183073828029534	-0.240367746052868\\
0.205651230834863	-0.303841967820631\\
0.231012970008314	-0.384241213126245\\
0.259502421139972	-0.486164271918028\\
0.291505306282517	-0.61549704682858\\
0.327454916287773	-0.779777935676805\\
0.367837977182865	-0.988646326976948\\
0.413201240011537	-1.25434290715009\\
0.464158883361283	-1.59211244938638\\
0.521400828799976	-2.01999730517917\\
0.585702081805658	-2.55641109053721\\
0.65793322465756	-3.21058556414027\\
0.739072203352571	-3.95178809308129\\
0.83021756813197	-4.62568000920885\\
0.932603346883218	-4.81387416383899\\
1.04761575278967	-3.92968148339554\\
1.176811952435	-1.89299413892576\\
1.32194114846604	0.74783148145724\\
1.48496826225448	3.51445941616203\\
1.66810053720008	6.21097250359736\\
1.87381742286036	8.79102359464338\\
2.104904144512	11.2594255399948\\
2.36448941264539	13.6339063631164\\
2.65608778294667	15.9329005380049\\
3.35160265093885	20.3648187363557\\
4.22924287438953	24.6477688779505\\
5.33669923120639	28.8390153222938\\
7.56463327554623	35.0265954643949\\
12.0450354025879	43.1831601927104\\
30.5385550883341	59.3863005254522\\
1000	119.999992877569\\
};
\addlegendentry{$s^2 + \frac{3\, s}{5} + 1$}

\addplot [color=mycolor4, line width=1.0pt]
  table[row sep=crcr]{%
0.01	-12.0411998265593\\
1000	-12.0411998265593\\
};
\addlegendentry{$\frac{1}{4}$}

\end{axis}

\begin{axis}[%
width=0.856\linewidth,
height=0.209\linewidth,
at={(0\linewidth,0\linewidth)},
scale only axis,
xmode=log,
xmin=0.01,
xmax=1000,
xminorticks=true,
xlabel style={font=\color{white!15!black}},
xlabel={$\text{frequency }\omega\text{, rad/s}$},
ymin=-400,
ymax=200,
ytick={-405, -360, -315, -270, -225, -180, -135,  -90,  -45,    0,   45,   90,  135,  180},
ylabel style={font=\color{white!15!black}},
ylabel={$\angle{\text{H(j}\omega\text{)}}\text{, deg}$},
axis background/.style={fill=white},
axis x line*=bottom,
axis y line*=left,
xmajorgrids,
xminorgrids,
ymajorgrids,
scaled ticks=false,tick label style={/pgf/number format/fixed}
]
\addplot [color=mycolor1, line width=1.0pt, forget plot]
  table[row sep=crcr]{%
0.01	-359.938272027531\\
0.0126185688306603	-359.922106688425\\
0.0141747416292673	-359.912499428661\\
0.0159228279334111	-359.901706885892\\
0.0178864952905735	-359.889582687071\\
0.0200923300256509	-359.87596232466\\
0.0225701971963382	-359.860660885465\\
0.0253536449397018	-359.843470484857\\
0.028480358684357	-359.824157363967\\
0.0319926713779749	-359.80245859967\\
0.0359381366380452	-359.778078367346\\
0.0403701725859671	-359.750683683887\\
0.0453487850812848	-359.719899542092\\
0.0509413801481661	-359.685303326264\\
0.0572236765935013	-359.646418370271\\
0.0642807311728466	-359.602706481\\
0.072208090183854	-359.553559197819\\
0.0811130830789735	-359.498287486664\\
0.0911162756115487	-359.436109467271\\
0.102353102189896	-359.366135631639\\
0.114975699539774	-359.287350812894\\
0.129154966501481	-359.198591879447\\
0.145082877849596	-359.098519719086\\
0.162975083462056	-358.985583479259\\
0.18307382802954	-358.857974146946\\
0.205651230834856	-358.713563231522\\
0.231012970008321	-358.54982031051\\
0.259502421139964	-358.363700103863\\
0.291505306282527	-358.151484866817\\
0.327454916287762	-357.908560029565\\
0.367837977182877	-357.629087994869\\
0.413201240011523	-357.305522787787\\
0.464158883361298	-356.927868994226\\
0.521400828799959	-356.482516243308\\
0.585702081805696	-355.950341548998\\
0.65793322465756	-355.303489765869\\
0.73907220335262	-354.499632784081\\
0.83021756813197	-353.471087961492\\
0.932603346883279	-352.102551596371\\
1.04761575278967	-350.18083357988\\
1.17681195243493	-347.265636724938\\
1.32194114846604	-342.292506786556\\
1.48496826225438	-332.008885083996\\
1.66810053720008	-303.278234915938\\
1.87381742286029	-240.756541808825\\
2.10490414451206	-209.276845070272\\
2.36448941264531	-198.255551633392\\
2.65608778294676	-193.031628417377\\
2.98364724028324	-190.005542698169\\
3.35160265093896	-188.025685434163\\
3.76493580679236	-186.622939556986\\
4.22924287438967	-185.572523081272\\
4.7508101621027	-184.753776205757\\
5.33669923120656	-184.096275497605\\
5.99484250318932	-183.556176725988\\
6.73415065775119	-183.104733053047\\
7.56463327554623	-182.722270397554\\
8.49753435908696	-182.394823489528\\
9.54548456661833	-182.112160217941\\
10.7226722201026	-181.866571844039\\
12.0450354025879	-181.652105745768\\
13.5304777457973	-181.46406401375\\
15.1991108295295	-181.298667110955\\
17.073526474706	-181.152822943505\\
19.1791026167252	-181.02396490113\\
21.5443469003179	-180.909935990256\\
24.2012826479444	-180.808904356664\\
27.1858824273284	-180.719300553818\\
30.5385550883351	-180.639770118275\\
34.3046928631481	-180.569137085891\\
38.5352859371068	-180.506375446093\\
43.2876128108296	-180.450586442503\\
48.6260158006557	-180.400980245007\\
54.6227721768425	-180.356860940904\\
61.3590727341349	-180.317614085345\\
68.9261210434963	-180.2826962558\\
77.4263682681172	-180.251626199606\\
86.974900261778	-180.223977266458\\
97.7009957299288	-180.199370891559\\
109.749876549306	-180.17747094876\\
123.284673944199	-180.157978832372\\
138.488637139389	-180.140629155426\\
155.567614393039	-180.125185974103\\
174.752840000771	-180.111439464619\\
196.304065004018	-180.099202991687\\
220.51307399031	-180.088310517682\\
278.255940220721	-180.069982906696\\
351.119173421526	-180.055459426443\\
443.062145758407	-180.043950233781\\
559.08101825125	-180.034829602989\\
705.480231071904	-180.027601762368\\
1000	-180.019472436454\\
};
\addplot [color=mycolor2, line width=1.0pt, forget plot]
  table[row sep=crcr]{%
0.01	-0.090373854426204\\
0.0112332403297803	-0.101519100583772\\
0.0126185688306603	-0.114038814242392\\
0.0141747416292682	-0.128102496435261\\
0.0159228279334111	-0.143900550142533\\
0.0178864952905741	-0.161646857334077\\
0.0200923300256502	-0.181581673544144\\
0.022570197196339	-0.203974879020137\\
0.025353644939701	-0.229129630254747\\
0.0284803586843579	-0.257386461043623\\
0.0319926713779738	-0.28912788817361\\
0.0359381366380464	-0.324783583503759\\
0.0403701725859658	-0.364836181621214\\
0.0453487850812863	-0.40982780050787\\
0.0509413801481645	-0.460367361807926\\
0.0572236765935013	-0.517138807406525\\
0.0642807311728424	-0.580910320163056\\
0.072208090183854	-0.652544668819928\\
0.0811130830789682	-0.733010810318234\\
0.0911162756115487	-0.823396896929069\\
0.102353102189903	-0.924924850597947\\
0.114975699539774	-1.03896668241042\\
0.129154966501489	-1.16706275063436\\
0.145082877849596	-1.31094216560905\\
0.162975083462067	-1.4725455626487\\
0.183073828029534	-1.65405047333562\\
0.205651230834863	-1.85789952847952\\
0.231012970008314	-2.08683171881283\\
0.259502421139972	-2.34391691669913\\
0.291505306282517	-2.632593815955\\
0.327454916287773	-2.95671136629689\\
0.367837977182865	-3.32057364849528\\
0.413201240011537	-3.728987934659\\
0.464158883361283	-4.18731537587847\\
0.521400828799976	-4.70152331714542\\
0.585702081805658	-5.27823760454206\\
0.65793322465756	-5.92479235421563\\
0.739072203352571	-6.64927341170461\\
0.83021756813197	-7.46055004300061\\
0.932603346883218	-8.36828715566912\\
1.04761575278967	-9.3829274489176\\
1.176811952435	-10.5156292817291\\
1.32194114846604	-11.7781417892179\\
1.48496826225448	-13.1825941596051\\
1.66810053720008	-14.7411716994927\\
1.87381742286036	-16.4656486851269\\
2.104904144512	-18.3667492170871\\
2.36448941264539	-20.4533155709618\\
2.65608778294667	-22.7312828657634\\
2.98364724028333	-25.2024930859187\\
3.35160265093885	-27.8634321364978\\
3.76493580679249	-30.7040366025754\\
4.22924287438953	-33.706778896636\\
4.75081016210285	-36.846276631549\\
5.99484250318932	-43.3977943244393\\
7.56463327554623	-50.0339028808779\\
8.4975343590864	-53.2740446216517\\
9.54548456661833	-56.4089444523826\\
10.7226722201033	-59.4059924509718\\
12.0450354025879	-62.2401007268698\\
13.5304777457982	-64.8940338721927\\
15.1991108295295	-67.3579941287641\\
17.0735264747072	-69.6286917017763\\
19.1791026167246	-71.7081454617614\\
21.5443469003186	-73.6024206557033\\
24.2012826479436	-75.3204478270693\\
27.1858824273293	-76.8730044438621\\
30.5385550883341	-78.271890681773\\
34.3046928631493	-79.5292972071695\\
38.5352859371055	-80.6573439669802\\
43.287612810831	-81.6677610523016\\
48.6260158006541	-82.5716816795177\\
54.6227721768443	-83.3795200586458\\
61.3590727341309	-84.1009112351034\\
68.9261210434963	-84.7446946047154\\
77.4263682681121	-85.3189270396727\\
86.974900261778	-85.8309151462782\\
97.7009957299225	-86.2872590486054\\
109.749876549306	-86.6939023123602\\
123.284673944207	-87.0561842916509\\
138.488637139389	-87.3788924075742\\
155.567614393049	-87.6663127517601\\
174.752840000771	-87.9222780345639\\
196.304065004024	-88.1502123337672\\
220.513073990302	-88.3531723974132\\
247.707635599169	-88.5338854529034\\
278.255940220711	-88.694783602694\\
312.571584968824	-88.8380349659717\\
351.119173421514	-88.965571770811\\
394.420605943768	-89.0791156233965\\
443.062145758392	-89.1802001876314\\
497.702356433218	-89.2701915052564\\
559.081018251231	-89.3503061772138\\
628.029144183417	-89.4216276139955\\
705.480231071857	-89.4851205478514\\
792.482898353912	-89.5416439841792\\
890.215085445036	-89.5919627539154\\
1000	-89.6367578137874\\
};
\addplot [color=mycolor3, line width=1.0pt, forget plot]
  table[row sep=crcr]{%
0.01	0.343804931539466\\
0.0112332403297803	0.386213242458126\\
0.0126185688306603	0.433855233481495\\
0.0141747416292682	0.487377894998616\\
0.0159228279334111	0.547508655072193\\
0.0178864952905746	0.615065570225141\\
0.0200923300256509	0.690968886304347\\
0.0225701971963397	0.776254185859983\\
0.0253536449397018	0.872087385129959\\
0.0284803586843589	0.979781904557086\\
0.0319926713779728	1.10081841716124\\
0.0359381366380452	1.23686768662375\\
0.0403701725859645	1.38981715243975\\
0.0453487850812848	1.56180211844\\
0.0509413801481628	1.75524267567494\\
0.0572236765935013	1.97288787341449\\
0.0642807311728424	2.21786919002321\\
0.072208090183854	2.49376611817473\\
0.0811130830789682	2.80468776952077\\
0.0911162756115487	3.15537597772186\\
0.102353102189903	3.55133767205882\\
0.114975699539774	3.99901767199145\\
0.129154966501489	4.50602809017352\\
0.145082877849596	5.08145814679651\\
0.162975083462067	5.73629989374299\\
0.18307382802954	6.48404363191921\\
0.205651230834869	7.34152597054464\\
0.231012970008321	8.33016104712763\\
0.259502421139981	9.47776493794362\\
0.291505306282527	10.8213197373077\\
0.327454916287762	12.4112644965761\\
0.367837977182853	14.3183367225391\\
0.413201240011523	16.6447994591085\\
0.464158883361268	19.5434200137645\\
0.521400828799959	23.2504224610713\\
0.585702081805658	28.1434827263735\\
0.65793322465756	34.8407561496499\\
0.739072203352571	44.3404266981106\\
0.83021756813197	58.0436746693979\\
0.932603346883218	76.8964217228337\\
1.04761575278967	98.8170084164814\\
1.176811952435	118.594692140941\\
1.32194114846604	133.30335940268\\
1.48496826225448	143.523628559224\\
1.66810053720008	150.686932701112\\
1.87381742286042	155.881368987926\\
2.10490414451206	159.789416023638\\
2.36448941264547	162.827332329691\\
2.65608778294676	165.253926976981\\
2.98364724028343	167.235418434992\\
3.35160265093874	168.882318919957\\
3.76493580679236	170.270586446359\\
4.22924287438939	171.454065468379\\
4.7508101621027	172.472023022629\\
5.33669923120621	173.353848877205\\
5.99484250318932	174.122067654918\\
6.73415065775075	174.794317666388\\
7.56463327554623	175.384680632897\\
8.4975343590864	175.904593966337\\
9.54548456661833	176.36348888471\\
10.7226722201033	176.769245031117\\
12.0450354025879	177.128520160473\\
13.5304777457982	177.446993417226\\
15.1991108295295	177.729547957407\\
17.0735264747072	177.980410381672\\
19.1791026167252	178.203258981078\\
21.5443469003193	178.40130914193\\
24.2012826479444	178.577381780289\\
27.1858824273302	178.733958981171\\
30.5385550883351	178.873229844753\\
34.3046928631481	178.997128723155\\
38.5352859371042	179.107367454912\\
43.2876128108296	179.205462794767\\
48.6260158006525	179.292759943099\\
54.6227721768425	179.370452867268\\
61.3590727341309	179.439601952356\\
68.9261210434963	179.501149404698\\
77.4263682681121	179.555932746423\\
86.974900261778	179.604696674977\\
97.7009957299225	179.648103512483\\
109.749876549306	179.686742431654\\
123.284673944207	179.721137615099\\
138.488637139389	179.751755480948\\
155.567614393049	179.779011088493\\
174.752840000771	179.803273821733\\
196.304065004031	179.824872435659\\
220.51307399031	179.844099539156\\
247.707635599177	179.861215579141\\
278.255940220721	179.876452382629\\
312.571584968834	179.890016306652\\
351.119173421503	179.902091040033\\
394.420605943755	179.912840095943\\
497.702356433201	179.930927411951\\
628.029144183417	179.945261227561\\
792.482898353912	179.956620494967\\
1000	179.96562250204\\
};
\addplot [color=mycolor4, line width=1.0pt, forget plot]
  table[row sep=crcr]{%
0.01	0\\
1000	0\\
};
\end{axis}
\end{tikzpicture}%
\caption{a bode multi example output.}
\label{fig:bode_multi_docs}
\end{figure}

\hypertarget{pole-docs}{%
\section{\texorpdfstring{\mintinline[]{text}{pole}
docs}{ docs}}\label{pole-docs}}

The following is the source code. Executing the cell writes the source
file!

\begin{minted}[]{matlab}
%%file ../pole.m
function out = pole(sys)

out = roots(cell2mat(sys.Den));
\end{minted}

\begin{minted}[]{text}
Created file '/Users/picone/code/matlab-rico/pole.m'.
\end{minted}

\hypertarget{usage-and-examples-2}{%
\subsection{Usage and examples}\label{usage-and-examples-2}}

\hypertarget{zero-docs}{%
\section{\texorpdfstring{\mintinline[]{text}{zero}
docs}{ docs}}\label{zero-docs}}

The following is the source code. Executing the cell writes the source
file!

\begin{minted}[]{matlab}
%%file ../zero.m
function out = zero(sys)

out = roots(cell2mat(sys.Num));
\end{minted}

\begin{minted}[]{text}
Created file '/Users/picone/code/matlab-rico/zero.m'.
\end{minted}

\hypertarget{usage-and-examples-3}{%
\subsection{Usage and examples}\label{usage-and-examples-3}}

\hypertarget{tf2latex-docs}{%
\section{\texorpdfstring{\mintinline[]{text}{tf2latex}
docs}{ docs}}\label{tf2latex-docs}}

The following is the source code. Executing the cell writes the source
file!

\begin{minted}[]{matlab}
%%file ../tf2latex.m

function out = tf2latex(sys)
% TF2LATEX  converts tf model to LaTeX code
%   TEXT = TF2LATEX(SYS) converts the
%   tf model SYS to LaTeX text.
%   Dependencies: 
%     - toolboxes
%       - Control Systems
%       - Symbolic

syms s

num = sys.Numerator;
den = sys.Denominator;

out = latex(...
  poly2sym(...
    cell2mat(num),...
    s...
  )/...
  poly2sym(...
    cell2mat(den),...
    s...
  )...
);
\end{minted}

\hypertarget{usage-and-examples-4}{%
\subsection{Usage and examples}\label{usage-and-examples-4}}
